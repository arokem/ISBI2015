\documentclass[3p]{elsarticle}

\usepackage{hyperref}

\journal{NeuroImage}

%%%%%%%%%%%%%%%%%%%%%%%
%% Elsevier bibliography styles
%%%%%%%%%%%%%%%%%%%%%%%
%% To change the style, put a % in front of the second line of the current style and
%% remove the % from the second line of the style you would like to use.
%%%%%%%%%%%%%%%%%%%%%%%

%% Numbered
%\bibliographystyle{model1-num-names}

%% Numbered without titles
%\bibliographystyle{model1a-num-names}

%% Harvard
\bibliographystyle{model2-names}
\biboptions{authoryear}

%% Vancouver numbered
%\usepackage{numcompress}\bibliographystyle{model3-num-names}

%% Vancouver name/year
%\usepackage{numcompress}\bibliographystyle{model4-names}\biboptions{authoryear}

%% APA style
%\bibliographystyle{model5-names}\biboptions{authoryear}

%% AMA style
%\usepackage{numcompress}\bibliographystyle{model6-num-names}

%% `Elsevier LaTeX' style
\bibliographystyle{elsarticle-num}
%%%%%%%%%%%%%%%%%%%%%%%

\begin{document}


\section{Models}

\subsection{Scherrer (DIAMOND)} 
DIAMOND models the set of tissue compartments in each voxel by a finite sum of unimodal continuous distributions of diffusion tensors. This corresponds to an hybrid tissue model that combines biophysical and statistical modeling. 
As described in \citep{scherrer_mrm2015}, the DW  signal $S_k$ for a gradient vector $\mathbf{g}_k$ and b-value $b_k$ is modeled by: 
$S_k = S_0 \big[  \sum_{j=0}^{N} f_j  {\left( 1+ \frac{b_k \mathbf{g}_k^T  \mathbf{D}^0_j  \mathbf{g}_k }{\kappa_j}  \right)}^{-\kappa_j} \big]$,
where $S_0$ is the non-attenuated signal, $N$ is the number of compartments and $\kappa_j$ and $ \mathbf{D}^0_j$ are respectively the concentration and the expectation of the j$^\mathrm{th}$ continuous tensor distribution.
DIAMOND enables  assessment of compartment-specific diffusion characteristics such as the compartment FA (cFA), the compartment RD (cRD) and the compartment MD (cMD). It also provides a novel measure of microstructural heterogeneity for each compartment.

The estimation of a continuous distribution of diffusion tensors requires DW data acquired with same timing parameters $\delta$ and $\Delta$ \citep{scherrer_mrm2015}. Therefore, we fitted one DIAMOND model for each \{$\delta$, $\Delta$\} group  (\textit{i.e.}, for each TE group), leading to $12$ DIAMOND models. One shell was missing in each TE group; we predicted its signal using the corresponding DIAMOND model. The model estimation was achieved as follows. We first computed the mean  and standard deviation  of $S_0$ ($\mu_{S_0}$ and $\sigma_{S_0}$) within each TE group and discarded DW-signals whose intensity were larger than $\mu_{S_0} + 3\sigma_{S_0}$ (simple artefact correction). We then estimated DIAMOND parameters as described in \citet{scherrer_mrm2015}, considering Gaussian noise and cylindrical anisotropic compartments. 
For the genu we considered a model with one freely diffusing and one anisotropic compartment; for the fornix we considered a model with  one freely diffusing compartment and two anisotropic compartments.


\section*{References}
\bibliography{mybibfile}

\end{document}